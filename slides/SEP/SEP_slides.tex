\documentclass{beamer}
\usetheme{Warsaw}
\usepackage[utf8]{inputenc}
\usepackage{hyperref}
\usepackage{xcolor}
\usepackage{multimedia}
\usepackage[linesnumbered,ruled,vlined]{algorithm2e}
\usepackage{physics}
\usepackage{float}
\setbeamertemplate{footline}[frame number] 

\newcommand{\snote}[1]{\textcolor{blue}{[*** Jamie : #1 ***]}}

\title[SEP]{Inner Approximations and a NISQ Algorithm for the Quantum Separability Problem}
\subtitle{\url{https://github.com/ankith-mohan/SEP}}
\author{
    Jamie Sikora
    \And
    Ankith Mohan
}
\institute{
    Virginia Tech
}
\date{11/15/2021}

\SetKwInput{KwInput}{Input}
\SetKwInput{KwOutput}{Output}

\begin{document}

\frame{\titlepage}

\AtBeginSection[]
{
  \begin{frame}{Outline}
    \tableofcontents[currentsection]
  \end{frame}
}

%\begin{frame} 
%\snote{Have an slide for each bound. $\alpha$ own slide and motivation. UPPER BOUNDS. $\beta$ for each bound. Symmetric extensions, mention how this set is larger then SEP, and thus $\beta \geq \alpha$. Mention realignment, and how this also relaxes $SEP$ (lower priority). Mention PPT and how this relaxes SEP. Summary slide: $\alpha \leq \min\{ \beta_1, \ldots, \}$.  
%        LOWER BOUNDS. Discuss see saw (in general) whole slide. OG see saw (whole slide). OG2. Randomized. $\alpha \geq \max \{ \gamma_1, \ldots \}$. 
%        SQUARE ROOT (special, new). Explain the "square root bound" ($\delta'$). Explain how they are/are not SDPs. Define $\delta$. 
%        SUMMARY. $\delta, \gamma, \ldots, \leq \alpha \leq \delta', \beta_1, \ldots$. Explain now that we want to test to see how good we are approximating $\alpha$.
%        New CHUNK: how to quantify goodness of bounds.
%        New Chunk: NISQ. Explain going from Big SEP problem to small SEP problem (why these bounds are useful).}
%\end{frame}

\section{Problem}
    \begin{frame}{Problem Statement}
        \begin{itemize}
            \item[] Given a density matrix $\mathbf{\rho} \in Pos(\mathcal{A}\otimes\mathcal{B})$, we want to say whether or not $\mathbf{\rho}$ is entangled?
            \item[] Formally, define 
            $$
                SEP(\mathcal{X}:\mathcal{Y}) = \left\{\sum\limits_{i=1}^{k}p_{i}\ket{\psi_{i}}\bra{\psi_{i}} \otimes \ket{\phi_{i}}\bra{\phi_{i}}: \sum\limits_{i=1}^{k}p_{i} = 1, p_{i} \ge 0\ \forall\ i\right\}
            $$
            \item[] We want to compute
            $$
                \alpha = \sup\limits_{\mathbf{\rho}_{\alpha} \in SEP}\langle\mathbf{\rho}_{\alpha}, \mathbf{\Pi}\rangle
            $$
            \item[] This is shown to be an NP-hard problem\footnote{\cite{gurvits2004classical}}
        \end{itemize}
    \end{frame}    
    
\section{Upper Bounds}
    \begin{frame}{Symmetric Extensions}
        \begin{itemize}
            \item[] Instead of asking whether or not $\mathbf{\rho}$ is entangled?
            \item[] We ask whether $\mathbf{\rho}$ is symmetric extendible?
            \item[] We compute
            $$
                \beta_{k} = \sup\limits_{\mathbf{\rho}_{\beta_{k}} \in SymExt(k)}\langle\mathbf{\rho}_{\beta_{k}}, \Pi\rangle
            $$
            \item[] $\because SEP = \bigcap_{k=1}^{\infty}SymExt(k)$
            \item[] $\implies \beta_{1} \ge \beta_{2} \ge \dots \ge \beta_{k} \ge \alpha$
        \end{itemize}
    \end{frame}
    
    %\begin{frame}{Realignment}
    %    \begin{itemize}
    %        \item[] Find
    %        $$
    %            \beta_{r} = \sup\limits_{\mathbf{\rho}_{\beta_{r}} \in Rlmt}\langle\mathbf{\rho}_{\beta_{r}}, \Pi\rangle
    %        $$
    %    \end{itemize}
    %\end{frame}
    
    \begin{frame}{Summary}
        $$
             \alpha \le \{\beta_{r}, \beta_{1}, \dots, \beta_{k}\}
        $$
    \end{frame}
    
\section{Lower Bounds}
    \begin{frame}{See-saw}
        \begin{algorithm}[H]
            \KwInput{$\mathbf{\sigma}_{A} \in Pos(\mathcal{A}),\ \mathbf{\Pi}$} 
            \KwOutput{$\gamma$}
            \Repeat{convergence or maximum number of iterations}{
                $\mathbf{\sigma}_{B}: \sup\limits_{\mathbf{\sigma}_{B} \in Pos(\mathcal{B})}\langle\mathbf{\sigma}_{A}\otimes\mathbf{\sigma}_{B}, \mathbf{\Pi}\rangle$ \\
                $\mathbf{\sigma}_{A}: \sup\limits_{\mathbf{\sigma}_{A} \in Pos(\mathcal{A})}\langle\mathbf{\sigma}_{A}\otimes\mathbf{\sigma}_{B}, \mathbf{\Pi}\rangle$
            }
            $\gamma = \langle\mathbf{\sigma}_{A}\otimes\mathbf{\sigma}_{B}, \mathbf{\Pi}\rangle$
            \caption{see-saw}
        \end{algorithm}
    \end{frame}
    
    \begin{frame}{Maximally-mixed (MM) see-saw}
        $$
            \gamma_{MM} = \mbox{see-saw}\left(\frac{1}{\dim\mathcal{A}}\,\mathbb{I}, \mathbf{\Pi}\right)
        $$
    \end{frame}
    
    \begin{frame}{Uniform superposition (US) see-saw}
        $$
            \gamma_{US} = \mbox{see-saw}\left(\frac{1}{\dim\mathcal{A}}\,\sum\limits_{i}\ket{i}\bra{i}, \mathbf{\Pi}\right)
        $$
    \end{frame}
    
    \begin{frame}{Randomized see-saw}
        \begin{algorithm}[H]
            \KwInput{$\mathbf{\Pi},\,N_{rand}$} 
            \KwOutput{$\gamma_{rand}$}
            $\gamma_{list} \rightarrow \{\emptyset\}$ \\
            \For{$i \rightarrow 1:N_{rand}$}{
                Initialize $\mathbf{\sigma}_{A}$ to be a random matrix in $Pos(\mathcal{A})$ \\
                $\gamma_{list}(i) = \mbox{see-saw}(\mathbf{\sigma}_{A}, \mathbf{\Pi})$
            }
            $\gamma_{rand} = \max\{\gamma_{list}\}$
            \caption{random see-saw}
        \end{algorithm}
    \end{frame}
    
    \begin{frame}{Summary}
        $$
            \{\gamma_{MM}, \gamma_{US}, \gamma_{rand}\} \le \alpha
        $$
    \end{frame}
    
\section{Square root}
    \begin{frame}{Idea}
        $$
            \sup\limits_{\mathbf{\sigma}_{A}\in Pos(\mathcal{A}), \mathbf{\sigma}_{B}\in Pos(\mathcal{B})}\langle\mathbf{\sigma}_{A}\otimes\mathbf{\sigma}_{B}, \mathbf{\Pi}\rangle \le \sup\limits_{\mathbf{\sigma}_{A}\in Pos(\mathcal{A}), \mathbf{\sigma}_{B}\in Pos(\mathcal{B})}\langle\sqrt{\mathbf{\sigma}_{A}}\otimes\sqrt{\mathbf{\sigma}_{B}}, \mathbf{\Pi}\rangle
        $$
        \footnote{This inequality holds only when $\mathbf{\Pi}$ is a PSD matrix}
    \end{frame}
    
    \begin{frame}{Idea}
        $$
            \textcolor{red}{\sup\limits_{\mathbf{\sigma}_{A}\in Pos(\mathcal{A}), \mathbf{\sigma}_{B}\in Pos(\mathcal{B})}\langle\mathbf{\sigma}_{A}\otimes\mathbf{\sigma}_{B}, \mathbf{\Pi}\rangle} \le \sup\limits_{\mathbf{\sigma}_{A}\in Pos(\mathcal{A}), \mathbf{\sigma}_{B}\in Pos(\mathcal{B})}\langle\sqrt{\mathbf{\sigma}_{A}}\otimes\sqrt{\mathbf{\sigma}_{B}}, \mathbf{\Pi}\rangle
        $$
        \textcolor{red}{This is not SDP :(} \\
        \footnote{$\mathbf{\Pi}$ is a PSD matrix}
    \end{frame}
    
    \begin{frame}{Idea}
        $$
            \textcolor{red}{\sup\limits_{\mathbf{\sigma}_{A}\in Pos(\mathcal{A}), \mathbf{\sigma}_{B}\in Pos(\mathcal{B})}\langle\mathbf{\sigma}_{A}\otimes\mathbf{\sigma}_{B}, \mathbf{\Pi}\rangle} \le \textcolor{violet}{\sup\limits_{\mathbf{\sigma}_{A}\in Pos(\mathcal{A}), \mathbf{\sigma}_{B}\in Pos(\mathcal{B})}\langle\sqrt{\mathbf{\sigma}_{A}}\otimes\sqrt{\mathbf{\sigma}_{B}}, \mathbf{\Pi}\rangle}
        $$
        \textcolor{red}{This is not SDP :(}\hfill\textcolor{violet}{This is SDP :)} \\
        \footnote{This inequality only holds when $\mathbf{\Pi}$ is a PSD matrix}
    \end{frame}
    
    \begin{frame}{How do we solve this?}
        \begin{align*}
            \mu = \sup\limits_{\mathbf{W}\in Herm(\mathcal{A}\otimes\mathcal{B})}\langle\mathbf{W}, \mathbf{\Pi}\rangle \\
            \mbox{subject to} \\
            \left[\begin{array}{cc}
                \mathbf{\sigma}_{A_{\delta}}\otimes\mathbb{I}_{\dim\mathcal{A}} & \mathbf{W} \\
                \mathbf{W} & \mathbb{I}_{\dim\mathcal{B}}\otimes\mathbf{\sigma}_{B_{\delta}}
            \end{array}\right] &\ge 0 \\
            \mathbf{\sigma}_{A_{\delta}} &\ge 0 \\
            \Tr{\mathbf{\sigma}_{A_{\delta}}} &= 1 \\
            \mathbf{\sigma}_{B_{\delta}} &\ge 0 \\
            \Tr{\mathbf{\sigma}_{B_{\delta}}} &= 1
        \end{align*}
        $\sigma_{A_{\delta}}$ and $\sigma_{B_{\delta}}$ correspond to the optimal solution.
    \end{frame}
    
    \begin{frame}{New bound}
        \begin{itemize}
            \item[] 
            $$
                \langle\sigma_{A_{\delta}}\otimes\sigma_{B_{\delta}}, \mathbf{\Pi}\rangle \le \alpha \le \langle\sqrt{\mathbf{\sigma}_{A_{\delta}}}\otimes\sqrt{\mathbf{\sigma}_{B_{\delta}}}, \mathbf{\Pi}\rangle
            $$
            where
            $$
                \alpha = \sup\limits_{\mathbf{\rho}_{\alpha} \in SEP}\langle\mathbf{\rho}_{\alpha}, \mathbf{\Pi}\rangle
            $$
        \end{itemize}
    \end{frame}
    
    \begin{frame}{New bound}
        \begin{itemize}
            \item[] 
            $$
                \gamma_{sqrt} = \mbox{see-saw}(\mathbf{\sigma}_{A_{\delta}}, \mathbf{\Pi}) \le \alpha \le \beta_{sqrt} = \langle\sqrt{\mathbf{\sigma}_{A_{\delta}}}\otimes\sqrt{\mathbf{\sigma}_{B_{\delta}}}, \mathbf{\Pi}\rangle
            $$
            where
            $$
                \alpha = \sup\limits_{\mathbf{\rho}_{\alpha} \in SEP}\langle\mathbf{\rho}_{\alpha}, \mathbf{\Pi}\rangle
            $$
        \end{itemize}
        \footnote{This inequality only holds when $\mathbf{\Pi}$ is a PSD matrix.}
    \end{frame}
    
    \begin{frame}{Summary}
        $$
            \{\gamma_{MM}, \gamma_{US}, \gamma_{rand}, \gamma_{sqrt}\} \le \alpha \le \{\beta_{sqrt}, \beta_{r}, \beta_{1}, \dots, \beta_{k}\}
        $$
    \end{frame}
    
    \section{Quantification}
        \begin{frame}{Quantification}
            \begin{itemize}
                \item[] How well are we really approximating $\alpha$?
                \item[] Define ``Tightness of Bound" (TOB) as
                $$
                    ToB = \beta - \gamma
                $$
                where
                \item[] 
                $$
                    \beta = \min\{\beta_{sqrt}, \beta_{r}, \beta_{1}, \dots, \beta_{k}\}
                $$
                \item[]
                $$
                    \gamma = \max\{\gamma_{MM}, \gamma_{US}, \gamma_{rand}, \gamma_{sqrt}\}
                $$
            \end{itemize}
        \end{frame}
    
    \section{NSS}
        \begin{frame}{Hybrid density matrix ansatz}
            \begin{itemize}
                \item[] Alice: $\{\ket{\psi_{i}}\}_{i=1}^{N} \in \mathcal{A}$
                \item[] Bob: $\{\ket{\phi_{j}}\}_{j=1}^{M} \in \mathcal{B}$
                \item[]
                $$
                    \mathbf{X} = \sum\limits_{ijkl}\beta_{ijkl}\ket{\psi_{i}}\bra{\psi_{j}}\otimes\ket{\phi_{k}}\bra{\phi_{l}}
                $$
            \end{itemize}
        \end{frame}
        
        \begin{frame}{Original Problem}
            \begin{align*}
                \sup\langle\mathbf{X}, \mathbf{\Pi}\rangle \\
                \mbox{subject to} \\
                Tr(\mathbf{X}) &= 1 \\
                \mathbf{X} &\in SEP(\mathcal{A}:\mathcal{B})
            \end{align*}
        \end{frame}

        \begin{frame}{Lemma}
            $$
                \mathbf{\beta} \in Herm(\mathcal{A}' \otimes \mathcal{B}')
            $$
            where
            \begin{align*}
                dim(\mathcal{A}') &<< dim(\mathcal{A}), \\
                dim(\mathcal{B}') &<< dim(\mathcal{B})
            \end{align*}
            \pause
            \begin{block}{Lemma}
                $$
                    \mathbf{\beta} \in SEP(\mathcal{A}' \otimes \mathcal{B}') \implies \mathbf{X} \in SEP(\mathcal{A} \otimes \mathcal{B})
                $$
            \end{block}
        \end{frame}

        \begin{frame}{Problem (Update 1)}
            \begin{align*}
                \sup\langle\mathbf{X}, \mathbf{\Pi}\rangle \\
                \mbox{subject to} \\
                Tr(\mathbf{X}) &= 1 \\
                \mathbf{X} &\in SEP(\mathcal{A}:\mathcal{B}) \rightarrow \textcolor{orange}{\mathbf{\beta} \in SEP(\mathcal{A}':\mathbf{B}')}
            \end{align*}
        \end{frame}

        \begin{frame}{Objective function}
            $$
                \mathbf{\Pi} = \sum\limits_{x}\mathbf{C}_{x}\mathbf{U}_{x}
            $$
            \begin{align*}
                \langle\mathbf{X}, \mathbf{\Pi}\rangle &= \langle\sum\limits_{ijkl}\beta_{ijkl}\ket{\psi_{i}}\bra{\psi_{j}}\otimes\ket{\phi_{k}}\bra{\phi_{l}}, \sum\limits_{x}\mathbf{C}_{x}\mathbf{U}_{x}\rangle \\
                &= \sum\limits_{ijklx}\beta_{ijkl}\mathbf{C}_{x}\langle\bra{\psi_{j}}\bra{\phi_{l}}\mathbf{U}_{x}\ket{\psi_{i}}\ket{\phi_{k}}\rangle \\
                &= \sum\limits_{ijkl}\langle\beta_{ijkl}, \sum\limits_{x}\mathbf{C}_{x}d_{ijklx}\rangle \\
                &= \langle\mathbf{\beta}, \mathcal{D}\rangle
            \end{align*}
        \end{frame}
        
        \begin{frame}{Problem (Update 2)}
            \begin{align*}
                \sup\langle\mathbf{X}, \mathbf{\Pi}\rangle \rightarrow \textcolor{orange}{\sup\langle\mathbf{\beta}, \mathcal{D}\rangle} \\
                \mbox{subject to} \\
                Tr(\mathbf{X}) &= 1 \\
                \mathbf{X} &\in SEP(\mathcal{A}:\mathcal{B}) \rightarrow \textcolor{orange}{\mathbf{\beta} \in SEP(\mathcal{A}':\mathcal{B}')}
            \end{align*}
        \end{frame}

        \begin{frame}{Density matrix constraint}
            \begin{align*}
                Tr(\mathbf{X}) &= Tr(\sum\limits_{ijkl}\beta_{ijkl}\ket{\psi_{i}}\bra{\psi_{j}}\otimes\ket{\phi_{k}}\bra{\phi_{l}}) \\
                &= \sum\limits_{ijkl}\beta_{ijkl}\braket{\psi_{j}}{\psi_{i}}\braket{\phi_{l}}{\phi_{k}} \\
                &= \sum\limits_{ijkl}\langle\beta_{ijkl}, e_{ijkl}\rangle \\
                &= \langle\mathbf{\beta}, \mathcal{E}\rangle
            \end{align*}
        \end{frame}

        \begin{frame}{Modified Problem}
            \begin{align*}
                \sup\langle\mathbf{X}, \mathbf{\Pi}\rangle \rightarrow \textcolor{orange}{\sup\langle\mathbf{\beta}, \mathcal{D}\rangle} \\
                \mbox{subject to} \\
                Tr(\mathbf{X}) &= 1 \rightarrow \textcolor{orange}{\langle\mathbf{\beta}, \mathcal{E}\rangle = 1} \\
                \mathbf{X} &\in SEP(\mathcal{A}:\mathcal{B}) \rightarrow \textcolor{orange}{\mathbf{\beta} \in SEP(\mathcal{A}':\mathcal{B}')}
            \end{align*}
        \end{frame}

    \section{Results}
        \begin{frame}{Competing hypotheses}
            \begin{enumerate}
                \item (Jamie): $\gamma_{rand}$ should work better for smaller values of $d$. Intuitively, smaller values of $d$ correspond to a smaller number of baskets. So starting from a set of random starting points, it is highly likely that we are going to reach the global optimum.
                \item (Kishor): $\gamma_{rand}$ might not work well for smaller values of $d$ because here we will be dealing with a bad landscape. As the value of $d$ increases, the landscape becomes more smooth and we would expect to reach the optimum more easily.
            \end{enumerate}
        \end{frame}

\bibliographystyle{apalike}
\bibliography{refs}
    
\end{document}
